\documentclass[conference,harvard,brazil,english]{sbatex}
\usepackage[utf8]{inputenc}
\usepackage[T1]{fontenc}
\usepackage{amsmath}
\begin{document}
\selectlanguage{brazil}
\title{Automatização da correção de provas utilizando processamento digital de imagens}
\author{João Marcos C. de Lima Costa}{jmcosta944@gmail.com}
\address{Universidade Federal do Rio Grande do Norte
\\Departamento de Engenharia Elétrica \\
Natal, RN, Brasil}
\author[1]{Marcus Vinicius de Oliveira Brito}{mvobrito@gmail.com}
\maketitle
\selectlanguage{english}
\begin{abstract}
The presented technique sequence will process an answer sheet's photo - belonging to a standard model - intending to obtain the essential informations contained on the sheet: student's registration number, the answers and score. All the procedure described was embedded into Android platform for smartphones through the application Kremlin for automating test correction.
\end{abstract}
\keywords{photo, answer sheet, Android, correction}
\selectlanguage{brazil}
\begin{abstract}
A sequência de técnicas apresentada irá processar a foto de um gabarito - pertencente a um modelo único - 
no intuito de obter as informações essenciais contidas na folha: matrícula do aluno, suas respostas e pontuação.
Todo o procedimento descrito foi embarcado na plataforma Android para smartphones pelo aplicativo Kremlin para automatizar a correção de provas.
\end{abstract}
\keywords{foto, gabarito, Android, correção}

\section{Processamento digital da foto}
	A aplicação responsável por processar a foto tirada pelo usuário foi, inicialmente, desenvolvida na linguagem de programação C++, fazendo uso da biblioteca OpenCV. 
	Na execução da aplicação - via linha de comando do Ubuntu -, o usuário passa como parâmetro o nome do arquivo da foto, que está contida no mesmo diretório do programa. 
	
	
	A folha de respostas é padronizada no tamanho A4, contendo duas tabelas centralizadas: a superior serve para o aluno marcar os espaços correspondentes a sua matrícula, e a inferior serve para marcar as respostas.
	Existem legendas para orientar o aluno em ambas as tabelas: na matrícula, a primeira coluna indica que algarismo está sendo marcado (de 0 a 9), e o aluno deverá preencher a tabela da direita para esquerda, marcando apenas uma casa por coluna, de maneira análoga a própria escrita da matrícula; a tabela de respostas segue a lógica convencional, na qual a primeira coluna indica o número da questão (de 1 a 10), e a primeira linha indica a alternativa (de A a E).
	
	Dispostos nas extremidades (próximos aos vértices), estão quatro quadrados totalmente pretos, com 1 centímetro de lado cada. O quadrado superior esquerdo dista de 17.5 centímetros do quadrado superior direito, e 26.5 centímetros do quadrado inferior esquerdo. O quarto quadrado (inferior direito) está a 17.5 centímetros  do quadrado inferior esquerdo, e 26.5 centímetros abaixo do quadrado superior direito. Por fim, temos uma distribuição simétrica destas marcas.
\\ \\ \\ \\ \\ \\ \\ \\ \\ \\ \\ \\ \\ \\ \\

\subsection{Aquisição da imagem}
	A preparação da cena e o posicionamento correto da câmera condicionam o funcionamento correto do algoritmo.
	A câmera deve estar próxima da folha de modo que capture uma região ligeiramente interna as suas bordas laterais, de modo que não apareça o limite entre a borda e a superfície sobre a qual pousa a folha. Essa mesma estratégia se aplica a margem superior, fazendo que a foto só mostre a superfície na parte abaixo da margem inferior. Se houver interferência do fundo nas margens laterais e superior, a leitura será afetada, à não ser que este seja branco e não se perceba nenhum contorno ou sombra. Fica claro que essa configuração de posição só é possível com o smartphone paralelo à folha.
	
	Quanto à iluminação, o usuário deve evitar sombras fortes na cena, e prezar por uma iluminação homogênia. Vale salientar que o flash da câmera não pode ser utilizado, mesmo que a luz sobre a folha esteja fraca. 

\subsection{Melhoramento}

	Uma vez que a foto tenha sido capturada corretamente, o algoritmo irá importá-la em tons de cinza com o comando \begin{verbatim}
imread(argv[1],CV_LOAD_IMAGE_GRAYSCALE);
\end{verbatim} 
, visto que seria inadequado e desnecessário usar as informações de cores do modelo RGB.
	Depois dessa etapa, os processos de limiarização e borramento são realizados, um após o outro, 3 vezes seguidas, utilizando os comandos \begin{verbatim}
	blur(image,image,Size(2,2));
threshold(image,image,0,255,
CV_THRESH_BINARY | CV_THRESH_OTSU);
	\end{verbatim}, sendo o \emph{blur} responsável por borrar a imagem e o \emph{threshold} pela limiarização, utilizando o método de Otsu.
\subsubsection{Método de Otsu}
	Limiarizar a imagem consiste em fixar um limite e substituir o valor do pixel em questão por 0 ou 255 (preto ou branco, respectivamente). Esses dois valores podem variar, e essa operação é do tipo:
$$
 f(x,y) =
  \begin{cases}
   255       & \quad \text{se } f(x,y) > limite \\
     0 & \quad \text{se } f(x,y) < limite \\
  \end{cases}
$$

\end{document}
